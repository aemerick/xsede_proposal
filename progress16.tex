\documentclass[12pt,preprint]{aastex}
\textheight 9in
\textwidth 6.5in
\topmargin -0.5in
\oddsidemargin 0in 
\evensidemargin 0in

\title{Progress Report}

\begin{document}
\maketitle
We currently have an allocation of 4.37M SU on Stampede, expiring on 31 December 2015 (Grant TG-MCA99S024).  We have remaining half of the allocation, 2.16M SU on Stampede, which we are actively using, and expect to exhaust by the expiration date of our allocation.  In the last year five refereed papers were published based on work supported under this allocation, and two more are currently in final draft stage, based on completed thesis chapters defended in July 2016.

The bulk of the current year's allocation has been spent on production runs studying the effect of gravity in a fully-developed galactic fountain with high midplane resolution, and computing self-gravitating cloud collapse models in zoom-in regions.  The first paper based on this work has been published (Iba\~nez-Mej\'{\i}a, Mac Low, Klessen, \& Baczynski 2016), and two more are in prep based on the last two chapters of Iba\~nez-Mej\'{\i}a's completed thesis. The first paper demonstrates that external supernova explosions alone drive insufficiently strong turbulence to explain observed molecular cloud dynamics, while gravitational collapse results in dynamics consistent with the observations. This provides evidence in support of the paradigm that molecular clouds are hierarchically collapsing objects whose lives are terminated by star formation, rather than equilibrium structures.


Turning to protoplanetary disks, we used two-dimensional models with the Pencil code to study the behavior of migrating, massive planets in non-isothermal disks (Richert et al. 2015; XSEDE acknowledgment unfortunately missing).  We found that relaxing the assumption of isothermality resulted in turbulence driven by the passage of the planet. Apparently the tidal shock raised by the planet heats the gas sufficiently to result in buoyant motions radially for planets with masses exceeding about 10 times that of Jupiter.  The resulting turbulence increases the diffusion rate of angular momentum in the disk, potentially causing the gas in the disk to quickly accrete once any embedded planet reaches the critical mass.  Three dimensional models of this process have been submitted for publication (Lyra et al. 2015) using the first author's XSEDE allocation.



\end{document}