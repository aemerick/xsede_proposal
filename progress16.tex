\documentclass[12pt,preprint]{aastex}
\textheight 9in
\textwidth 6.5in
\topmargin -0.5in
\oddsidemargin 0in 
\evensidemargin 0in

\title{Progress Report}

\begin{document}
\maketitle
We currently have an allocation of 4.37M SU on Stampede, expiring on 31 December 2015 (Grant TG-MCA99S024).  We have remaining half of the allocation, 2.16M SU on Stampede, which we are actively using, and expect to exhaust by the expiration date of our allocation.  In the last year five refereed papers were published based on work supported under this allocation, and two more are currently in final draft stage, based on completed thesis chapters defended in July 2016.

The bulk of the current year's allocation has been spent on production runs studying the effect of gravity in a fully-developed galactic fountain with high midplane resolution, and computing self-gravitating cloud collapse models in zoom-in regions. These zoom-in models also provide the initial conditions for the stellar cluster formation simulations proposed here. The first paper based on this work has been published (Iba\~nez-Mej\'{\i}a, Mac Low, Klessen, \& Baczynski 2016), and two more are in prep based on the last two chapters of Iba\~nez-Mej\'{\i}a's completed thesis. The first paper demonstrates that external supernova explosions alone drive insufficiently strong turbulence to explain observed molecular cloud dynamics, while gravitational collapse results in dynamics consistent with the observations. This provides strong evidence in support of the paradigm that molecular clouds are hierarchically collapsing objects whose lives are terminated by star formation, rather than equilibrium structures.  

We also studied the stripping of satellite galaxies by ram pressure as they orbit through the hot, gaseous halo of their host. We found that in 3D, even if we include supernova feedback, the stripping is insufficiently strong to explain the rather prompt cutoffs in star formation observed in the star formation histories of small Milky Way satellites.  This implies that some other mechanism must have acted to strip the gas so quickly. Tidal forces from either the host or other satellites are a major possibility for this second mechanism.

CoPI Emerick used Enzo, the code proposed for use in our dwarf galaxy simulations, to study warm (10$^5$~K) gas in galaxy clusters.  Although this gas is hard to observe in emission, it can be studied in absorption of the Ly$\alpha$ line.  It was found that warm gas far from the center typically came from material falling into the cluster from large-scale intergalactic filaments, but warm gas close to the center tended instead to be gas stripped from cluster galaxies.  A corresponding bimodality in Ly$\alpha$ absorption depths appears to be seen in the observations.

Turning to protoplanetary disks, the PI collaborated on the use of three-dimensional models with the Pencil code to continue our study of the behavior of migrating, massive planets in non-isothermal disks (Lyra et al. 2016).  We found that our 3D confirm the basic picture that  the tidal density perturbations raised by planets with masses exceeding about 10 times that of Jupiter shocks and  heats the gas. The 3D structure resembles a tidal bore, with material lifting off the plane of the gas and then crashing back down in a turbulent flow.  The resulting turbulence increases the diffusion rate of angular momentum in the disk at least in regions near the spirals. The accompanying heating looks sufficient to explain the observed large pitch angles of spirals in protoplanetary disks.



\end{document}

A. EmerickM. LowJ. GrcevichA. Gatto. GAS LOSS BY RAM PRESSURE STRIPPING AND INTERNAL FEEDBACK FROM LOW-MASS MILKY WAY SATELLITES. ApJ. 2. 148. http://dx.doi.org/10.3847/0004-637X/826/2/148. 2016.
× Remove
W. LyraA. RichertA. BoleyN. TurnerM. Mac LowS. OkuzumiM. Flock. ON SHOCKS DRIVEN BY HIGH-MASS PLANETS IN RADIATIVELY INEFFICIENT DISKS. II. THREE-DIMENSIONAL GLOBAL DISK SIMULATIONS. ApJ. 2. 102. http://dx.doi.org/10.3847/0004-637X/817/2/102. 2016.
× Remove
J. Ibáñez-MejíaM. Mac LowR. KlessenC. Baczynski. Gravitational Contraction Versus Supernova Driving and the Origin of the Velocity Dispersion-Size Relation in Molecular Clouds. Astrophys. J.. 824. 41 (15 pp.). http://iopscience.iop.org/article/10.3847/0004-637X/824/1/41/meta;jsessionid=3177649B17C5D5F9CC8171ACE419B911.c1.iopscience.cld.iop.org. 2016.
× Remove
J. StaffO. De MarcoD. MacdonaldP. GalavizJ. PassyR. IaconiM. Low. Hydrodynamic simulations of the interaction between an AGB star and a main-sequence companion in eccentric orbits. Mon. Not. R. Astron. Soc.. 4. 3511-3525. http://dx.doi.org/10.1093/mnras/stv2548. 2015.
× Remove
A. EmerickG. BryanM. Putman. Warm Gas in and Around Simulated Galaxy Clusters as Probed by Absorption Lines. Monthly Notices of the Royal Astronomical Society. . . . 2015.
Add Publications
PreviousNext