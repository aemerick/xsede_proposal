% set \longtrue to make a more detailed CV.
% set \longfalse to make a shorter CV
\newif\iflong
\longtrue

\newif\iflongexperience

\iflong
	\longexperiencetrue
\fi
\longfalse

%%%%%%%%%%%%%%%%%%%%%%%%%%%%%%%%%%%%%%%%%%%%%%%%%%%%%%%%%%%%%%%%%%%%%%%%
%%%%%%%%%%%%%%%%%%%%%% Simple LaTeX CV Template %%%%%%%%%%%%%%%%%%%%%%%%
%%%%%%%%%%%%%%%%%%%%%%%%%%%%%%%%%%%%%%%%%%%%%%%%%%%%%%%%%%%%%%%%%%%%%%%%
\documentclass[10pt]{article}

% for Roman serrif font
%\usepackage{times}
%\renewcommand{\familydefault}{\sfdefault}
% This is a helpful package that puts math inside length specifications
\usepackage{calc}
\usepackage{comment}

% Simpler bibsection for CV sections
% (thanks to natbib for inspiration)
\makeatletter
\newlength{\bibhang}
\setlength{\bibhang}{1em} %1em}
\newlength{\bibsep}
 {\@listi \global\bibsep\itemsep \global\advance\bibsep by\parsep}
\newenvironment{bibsection}%
        {\begin{enumerate}{}{%
%        {\begin{list}{}{%
       \setlength{\leftmargin}{\bibhang}%
       \setlength{\itemindent}{-\leftmargin}%
       \setlength{\itemsep}{\bibsep}%
       \setlength{\parsep}{\z@}%
        \setlength{\partopsep}{0pt}%
        \setlength{\topsep}{0pt}}}
        {\end{enumerate}\vspace{-.6\baselineskip}}
%        {\end{list}\vspace{-.6\baselineskip}}
\makeatother

% Layout: Puts the section titles on left side of page
\reversemarginpar




%
%         PAPER SIZE, PAGE NUMBER, AND DOCUMENT LAYOUT NOTES:
%
% The next \usepackage line changes the layout for CV style section
% headings as marginal notes. It also sets up the paper size as either
% letter or A4. By default, letter was used. If A4 paper is desired,
% comment out the letterpaper lines and uncomment the a4paper lines.
%
% As you can see, the margin widths and section title widths can be
% easily adjusted.
%
% ALSO: Notice that the includefoot option can be commented OUT in order
% to put the PAGE NUMBER *IN* the bottom margin. This will make the
% effective text area larger.
%
% IF YOU WISH TO REMOVE THE ``of LASTPAGE'' next to each page number,
% see the note about the +LP and -LP lines below. Comment out the +LP
% and uncomment the -LP.
%
% IF YOU WISH TO REMOVE PAGE NUMBERS, be sure that the includefoot line
% is uncommented and ALSO uncomment the \pagestyle{empty} a few lines
% below.
%

%% Use these lines for letter-sized paper
\usepackage[paper=letterpaper,
            %includefoot, % Uncomment to put page number above margin
            marginparwidth=1.2in,     % Length of section titles
            marginparsep=.05in,       % Space between titles and text
            margin=1in,               % 1 inch margins
            includemp]{geometry}

%% Use these lines for A4-sized paper
%\usepackage[paper=a4paper,
%            %includefoot, % Uncomment to put page number above margin
%            marginparwidth=30.5mm,    % Length of section titles
%            marginparsep=1.5mm,       % Space between titles and text
%            margin=25mm,              % 25mm margins
%            includemp]{geometry}

%% More layout: Get rid of indenting throughout entire document
\setlength{\parindent}{0in}

\usepackage[shortlabels]{enumitem}

%% Reference the last page in the page number
%
% NOTE: comment the +LP line and uncomment the -LP line to have page
%       numbers without the ``of ##'' last page reference)
%
% NOTE: uncomment the \pagestyle{empty} line to get rid of all page
%       numbers (make sure includefoot is commented out above)
%
\usepackage{fancyhdr,lastpage}
\pagestyle{fancy}
%\pagestyle{empty}      % Uncomment this to get rid of page numbers
\fancyhf{}\renewcommand{\headrulewidth}{0pt}
\fancyfootoffset{\marginparsep+\marginparwidth}
\newlength{\footpageshift}
\setlength{\footpageshift}
          {0.5\textwidth+0.5\marginparsep+0.5\marginparwidth-2in}
\lfoot{\hspace{\footpageshift}%
       \parbox{4in}{\, \hfill %
                    \arabic{page} of \protect\pageref*{LastPage} % +LP
%                    \arabic{page}                               % -LP
                    \hfill \,}}

% Finally, give us PDF bookmarks
\usepackage{color,hyperref}
\definecolor{darkblue}{rgb}{0.0,0.0,0.9}
\hypersetup{colorlinks,breaklinks,
            linkcolor=darkblue,urlcolor=darkblue,
            anchorcolor=darkblue,citecolor=darkblue}

%%%%%%%%%%%%%%%%%%%%%%%% End Document Setup %%%%%%%%%%%%%%%%%%%%%%%%%%%%
% =================================================================================================%

%%%%%%%%%%%%%%%%%%%%%%%%%%% Helper Commands %%%%%%%%%%%%%%%%%%%%%%%%%%%%

% The title (name) with a horizontal rule under it
% (optional argument typesets an object right-justified across from name
%  as well)
%
% Usage: \makeheading{name}
%        OR
%        \makeheading[right_object]{name}
%
% Place at top of document. It should be the first thing.
% If ``right_object'' is provided in the square-braced optional
% argument, it will be right justified on the same line as ``name'' at
% the top of the CV. For example:
%
%       \makeheading[\emph{Curriculum vitae}]{Your Name}
%
% will put an emphasized ``Curriculum vitae'' at the top of the document
% as a title. Likewise, a picture could be included:
%
%   \makeheading[\includegraphics[height=1.5in]{my_picutre}]{Your Name}
%
% the picture will be flush right across from the name.
\newcommand{\makeheading}[2][]%
        {\hspace*{-\marginparsep minus \marginparwidth}%
         \begin{minipage}[t]{\textwidth+\marginparwidth+\marginparsep}%
             {\large \bfseries #2 \hfill #1}\\[-0.15\baselineskip]%
                 \rule{\columnwidth}{1pt}%
         \end{minipage}}

% The section headings
%
% Usage: \section{section name}
\renewcommand{\section}[1]{\pagebreak[3]%
    \hyphenpenalty=10000%
    \vspace{1.3\baselineskip}%
    \phantomsection\addcontentsline{toc}{section}{#1}%
    \noindent\llap{\scshape\smash{\parbox[t]{\marginparwidth}{\raggedright #1}}}%
    \vspace{-\baselineskip}\par}

% An itemize-style list with lots of space between items
\newenvironment{outerlist}[1][\enskip\textbullet]%
        {\begin{itemize}[#1,leftmargin=*]}{\end{itemize}%
         \vspace{-.6\baselineskip}}

% An environment IDENTICAL to outerlist that has better pre-list spacing
% when used as the first thing in a \section
\newenvironment{lonelist}[1][\enskip\textbullet]%
        {\begin{list}{#1}{%
        \setlength{\partopsep}{0pt}%
        \setlength{\topsep}{0pt}}}
        {\end{list}\vspace{-.6\baselineskip}}

% An itemize-style list with little space between items
\newenvironment{innerlist}[1][\enskip\textbullet]%
        {\begin{itemize}[#1,leftmargin=*,parsep=0pt,itemsep=0pt,topsep=0pt,partopsep=0pt]}
        {\end{itemize}}

% An environment IDENTICAL to innerlist that has better pre-list spacing
% when used as the first thing in a \section
\newenvironment{loneinnerlist}[1][\enskip\textbullet]%
        {\begin{itemize}[#1,leftmargin=*,parsep=0pt,itemsep=0pt,topsep=0pt,partopsep=0pt]}
        {\end{itemize}\vspace{-.6\baselineskip}}

% To add some paragraph space between lines.
% This also tells LaTeX to preferably break a page on one of these gaps
% if there is a needed pagebreak nearby.
\newcommand{\blankline}{\quad\pagebreak[3]}
\newcommand{\halfblankline}{\quad\vspace{-0.5\baselineskip}\pagebreak[3]}

% Uses hyperref to link DOI
\newcommand\doilink[1]{\href{http://dx.doi.org/#1}{#1}}
\newcommand\doi[1]{doi:\doilink{#1}}

% For \url{SOME_URL}, links SOME_URL to the url SOME_URL
\providecommand*\url[1]{\href{#1}{#1}}
% Same as above, but pretty-prints SOME_URL in teletype fixed-width font
\renewcommand*\url[1]{\href{#1}{\texttt{#1}}}

% For \email{ADDRESS}, links ADDRESS to the url mailto:ADDRESS
\providecommand*\email[1]{\href{mailto:#1}{#1}}
% Same as above, but pretty-prints ADDRESS in teletype fixed-width font
%\renewcommand*\email[1]{\href{mailto:#1}{\texttt{#1}}}

%\providecommand\BibTeX{{\rm B\kern-.05em{\sc i\kern-.025em b}\kern-.08em
%    T\kern-.1667em\lower.7ex\hbox{E}\kern-.125emX}}
%\providecommand\BibTeX{{\rm B\kern-.05em{\sc i\kern-.025em b}\kern-.08em
%    \TeX}}
\providecommand\BibTeX{{B\kern-.05em{\sc i\kern-.025em b}\kern-.08em
    \TeX}}
\providecommand\Matlab{\textsc{Matlab}}




%%%%%%%%%%%%%%%%%%%%%%%% End Helper Commands %%%%%%%%%%%%%%%%%%%%%%%%%%%

%%%%%%%%%%%%%%%%%%%%%%%%% Begin CV Document %%%%%%%%%%%%%%%%%%%%%%%%%%%%

\begin{document}
\makeheading{Joshua E. Wall\\ \textnormal{Drexel University}\\ \textnormal{American Museum of Natural History}}

\section{Contact Information}

% NOTE: Mind where the & separators and \\ breaks are in the following
%       table.
%
% ALSO: \rcollength is the width of the right column of the table
%       (adjust it to your liking; default is 1.85in).
%
\newlength{\rcollength}\setlength{\rcollength}{1.4in}%
%
\begin{tabular}[t]{@{}p{\textwidth-\rcollength}p{\rcollength}}
\email{jew99@drexel.edu} \\
\end{tabular}

%\section{Objective}

%Insert text here if you want to
%\begin{innerlist}
%\item More information and auxiliary documents can be found at\\\url{http://www.tedpavlic.com/facjobsearch/}
%\end{innerlist}

\section{Research Interests}

My research focuses on the formation and early evolution of star clusters. Specifically I use the magnetohydrodynamics code Flash, coupled with collisional N-body codes, to study how stellar formation is effected by local stellar feedback in the form of winds, radiation and supernova. This allows me to study how feedback quenches star formation and the efficiency of converting interstellar gas into stars. I also study how gas ejection can disrupt young stellar clusters, gaining insight into why approximately 90\% of all stellar clusters are disrupted before the end of their embedded phase. Finally, I am also interested in the puzzle of multiple stellar populations in globular clusters, including how pressure confinement may contribute to the formation of later generations of stars in specific environments.

\section{Education}

\href{}{\textbf{Drexel University}},
\begin{outerlist}

\item[] Ph.D.,
		\href{}
			{Physics},
			\emph{Expected:} 2017/2018
			
\item[] M.S.,
		\href{}
			{Physics},
			2015
\end{outerlist}

\vspace{.1in}
\textbf{Austin Peay State University},
Minneapolis, MN
\begin{outerlist}

%        \begin{innerlist}
%        \item Thesis Topic: \emph{asdf}
%        \item Advisors:
%              \href{}
%                   {name} and
%              \href{http://www.biostat.umn.edu/~sudiptob/}
%                   {Sudipto Banerjee, Ph.D}
%        \end{innerlist}

\item[] B.S.,
        %\href{http://www.apsu.edu/}
             Physics,
             May 2013
        \begin{innerlist}
		\item Summa Cum Laude    
%Advisor : \href{http://www.apsu.edu/physics/facstaff}{Alex King, Ph.D.}

        \end{innerlist}
\end{outerlist}

% The 'iflong' 'fi' statements in the following are used to disable/enable
% descriptions of research experience when generating CV, depending on 
% need (see if statements at top of file)
\section{Research Experience}

\textbf{Research Assistant} \hfill {July 2015 - Present}

\begin{innerlist}

\item[] Department of Physics, Drexel University - Philadelphia, PA\\
        American Museum of Natural History - New York, NY\\
    Supervisors: Stephen McMillan, Ph.D and Mordecai-Mark Mac Low, Ph.D
%		\begin{innerlist}
%		\item Research utilizes large scale computational simulations to study
%		galaxy formation and evolution on all scales, with a focus on dwarf galaxies.
%		\item Currently implementing a chemodynamics method into the AMR, hydro code 
%		\textsc{Enzo} using star-by-star modeling and including the effects of supernovae,
%		stellar winds, cosmic rays, and full radiative transfer.
%		\item Utilize following code projects: \href{http://enzo-project.org/}{Enzo}, 
%		\href{http://yt-project.org/}{\textit{yt}}, \href{http://flash.uchicago.edu/site/}{FLASH}
%		\end{innerlist}
\end{innerlist}

\iflong
\textbf{Undergraduate Research Assistant} \hfill {Jan. 2012 - July. 2012}
\begin{innerlist}

\item[] Department of Physics and Astronomy, Austin Peay State Univesity - Clarksville, TN\\
		Supervisor: Justin Oelgoetz, Ph.D
		\iflong
		\begin{innerlist}
		\item Studying the thermodynamic properties of silicon nanotubes using computational methods.
		\end{innerlist}
		\fi
\end{innerlist}
\fi

\section{Talks}
\vspace{-.125in}
\begin{bibsection}

    \item \textbf{J. Wall}, S. McMillan, M.-M. Mac Low, "Simulating the Formation and Early Evolution of Star Clusters". MODEST NYC
    New York, NY (July 2016)
    
    \item \textbf{J. Wall}, S. McMillan, M.-M. Mac Low, "Coupling Hydrodynamics and N-body Codes Using a Gravity Bridge". RAMBODY workshop,
    Leiden, Netherlands (October 2015)

\end{bibsection}

\section{Conference Publications}
\vspace{-.125in}
\begin{bibsection}
  \item \textbf{J. Wall}, S. McMillan, M.-M. Mac Low, "Formation and Early Evolution of Star Clusters". Kobe, Japan, MODEST 2015-S
    (December 2015)


  \item \textbf{J. Wall}, S. McMillan, M.-M. Mac Low, "Coupling Magnetohydrodynamics and N-body Codes Using a Gravity Bridge". 
    IAU 2015, Honolulu, HI. (August 2015)
  
  
  \item \textbf{J. Wall}, S. McMillan, M.-M. Mac Low, "Coupling Magnetohydrodynamics and N-body Codes Using a Gravity Bridge". 
  MODEST 2015, Concepi\'{o}n, Chile. (March 2015)
\end{bibsection}

\section{Teaching Experience}
\begin{innerlist}
\item T.A. Physics 101: University Physics for Engineers A \hfill {Fall 2013,2014}
\item T.A. Physics 102: University Physics for Engineers B \hfill {Winter 2014,2015}
\item T.A. Physics 201: University Physics for Engineers C \hfill {Spring 2014,2015}
\end{innerlist}

\halfblankline

% Add a little space to nudge next ``Conference Publications'' marginpar
% down to make room for tall ``Submitted Journal Publications''
% marginpar. If there are enough submitted journal publications, this
% space will not be needed (and should be removed).
%\vspace{0.1in}

%\section{Papers in Preparation}
%\vspace{-.1in}
%\begin{bibsection}
%    \item Toomey, T.L., Erickson, D.J., Carlin, B.P., Lenk, K.M., {\bf Quick, H.S.}, and Harwood, E.M. ``Do neighborhood attributes moderate the relationship between alcohol establishment density and crime?"
%    \item {\bf Quick, H.}, Banerjee, S., and Carlin, B.P. ``Heteroscedastic variances in areally referenced temporal processes with an application to California asthma hospitalization data.''

%    \item {\bf Quick, H.}, Carlin, B.P., and Banerjee, S. ``Space-time Gaussian process modeling of temporal air pollution gradients."
%\end{bibsection}

\section{Awards}
Graduate Awards
\begin{innerlist}
\item Dean's Fellowship \hfill Fall 2013, 2 yr \\
	  Graduate College, Drexel University
\end{innerlist}

\halfblankline

Travel Awards
\begin{innerlist}
\item International Travel Award\hfill December 2015\\
MODEST 2015-S, Kobe, Japan

\item International Travel Award\hfill March 2015\\
MODEST 2015, Concepci\'{o}n, Chile

\end{innerlist}

\halfblankline

Research Grants
\begin{innerlist}
\item NASA Science Mission Directorate High Perfomance Computing Grant \hfill 2015-2016 \\

\end{innerlist}

\halfblankline
\iflong
Undergraduate Scholarships
\begin{innerlist}
\item NASA Tennessee Space Grant \hfill 2011-2012 \\
Austin Peay State University
\end{innerlist}
\fi


%\section{Organizations}
%\begin{innerlist}
%\item American Physical Society (APS); Society of Physics Students (SPS), Sigma Pi
%Sigma: Physics Honors Society
%\end{innerlist}


%\section{Skills}
%Computer Programming:
%\begin{innerlist}
%    \item C$++$, Fortran, Python, LaTex, R, UNIX shell scripting, 
%    GNU make
%\end{innerlist}

%\halfblankline


\section{References}
Stephen McMillan
\begin{innerlist}
\item[] Professor of Physics and Chair \hfill {Phone: 1-212-854-6837}\\
Department of Physics \hfill{E-mail: steve@physics.drexel.edu }\\
Drexel University
\end{innerlist}

\halfblankline

Mordecai-Mark Mac Low
\begin{innerlist}
\item[] Professor and Curator \hfill {Phone: 612-624-1699}\\
Department of Astrophysics \hfill{E-mail: mordecai@amnh.org}\\
American Museum of Natural History
\end{innerlist}

\halfblankline

Alex King
\begin{innerlist}
\item[] Professor and Chair \hfill {Phone: 612-624-3396}\\
Department of Physics and Astronomy \hfill{E-mail: kinga@apsu.edu}\\
Austin Peay State University
\end{innerlist}


\halfblankline

Justin Oelgoetz
\begin{innerlist}
\item[] Professor \hfill {Phone: 612-624-3396}\\
Department of Physics and Astronomy \hfill{E-mail: oelgoetzj@apsu.edu}\\
Austin Peay State University
\end{innerlist}


\end{document}
